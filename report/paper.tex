% Matthew Monaco
% Andy Sayler
% Landon Spear
% University of Colorado
% Networked Character devices
% Fall 2011

\documentclass[11pt,twocolumn]{article}

\usepackage[text={6in, 9in}, centering]{geometry}
\usepackage{graphicx}
\usepackage{url}

\bibliographystyle{plain}

\newenvironment{packed_enum}{
\begin{enumerate}
  \setlength{\itemsep}{1pt}
  \setlength{\parskip}{0pt}
  \setlength{\parsep}{0pt}
}{\end{enumerate}}

\newenvironment{packed_item}{
\begin{itemize}
  \setlength{\itemsep}{1pt}
  \setlength{\parskip}{0pt}
  \setlength{\parsep}{0pt}
}{\end{itemize}}

\begin{document}

\title{Networked Character Devices}
\author{Monaco, Matthew \and Sayler, Andrew \and Spear, Landon
  \\ \and University of Colorado}
\date{\today}

\maketitle

\begin{abstract}
Lorem ipsum
\end{abstract}

% Landon Sections

\section{Introduction}
\label{sec:introduction}

We did some stuff, we used some references \cite{ldd3}.

\section{Related Work}
\label{sec:relatedwork}

We are unique.

% Matt Sections

% Combine Architecture and System?
\section{Architecture}
\label{sec:architecture}

\section{System}
\label{sec:system}

\section{Implementation}
\label{sec:implementation}

% Andy Sections

\section{Results and Evaluation}
\label{sec:results}

The Network Character Device system is still a work in
progress. We will need to Expand the system beyond it's current state
before the NCD concept is ready for production-level use.

Non the less, we have gained many insights into the advantages such a
system might provide. We have also encountered the difficulties
implementing such a system into an existing OS like Linux entails.

\subsection{Current Status}
\label{sec:currentstatus}

As of this writing, we have completed a partial implementation of
the Network Character Device system. Current functionality includes:

\begin{packed_item}
\item Support for exporting a single character device from the server
\item Support for importing a single character device on the client
\item Client-side kernel-space Linux module implementation
\item Server-side user-space background process implementation
\item Support for open, close, read, and write calls
\item Basic Linux udev support for automatic node creation on client
\end{packed_item}

This set of core functionality provides a working proof-of-concept
Network Character Device system. This system fully supports the
export of a single character device from server to client over the
network and demonstrates the basic use and utility of such a system.

It should be noted, however, that some work is still required before
the Network Character device system can be considered a production
ready utility. Most notably, the following features are necessary for
a fully functional NCD system, but are not yet included in our
implementation:

\begin{packed_item}
\item Multi-device, multi-server client-side import support
\item Multi-device server-side export support
\item Support for exclusive use and protection of exported devices
\item Support for ioctl calls
\item Support for providing and obtaining metadata for exported
  devices
\item Support for ``advertisement'' of available exported devices
\item Support for integrating imported devices into other kernel
  subsystems (human interface subsystem, audio subsystem, etc)
\item Addition of ``NCD-admin'' utilities for managing and administering
  NCD system
\end{packed_item}

Section \ref{sec:futurework} provides some insight into the addition
of some of these expanded features.

\subsection{Advantages}
\label{sec:advantages}

\subsection{Challenges}
\label{sec:challenges}

\section{Future Work}
\label{sec:futurework}

\section{Conclusion}
\label{sec:conclusion}

\bibliography{refs}

\end{document}
